\documentclass[a4paper]{article}
\usepackage{geometry}
\geometry{left=2cm,right=2cm,top=2.5cm,bottom=2.5cm}
\author{Chunwei Yan}

\title{SVD++ 模型理解}
\usepackage{xeCJK}
\usepackage{graphicx}
\setCJKmainfont{SimSun}
%\setCJKmainfont{WenQuanYi Micro Hei Mono}
\begin{document}
    \maketitle
SVD++融合以下模型或元素:
\begin{enumerate}
\item Baseline model
\item Latent factor model
\item Implicit feedback
\end{enumerate}

\section{Baseline model}
\par 有一般情况下,可以想象,一些用户倾向于总给电影打高分/低分,类似的,一些电影也倾向于总得到高分/低分。体现在模型上,可以有两个参数$b_u, b_i$
\begin{itemize}
\item $b_u$是用户$u$相对于平均评分$\mu$的偏差
\item $b_i$是电影$i$相对于$\mu$的偏差
\end{itemize}
\par 那么最终的预测模型就是
\begin{equation}
    \hat{r}_{ui} = \mu + b_u + b_i
\end{equation}

\section{Latent factor model}
\par 利用非负矩阵分解的方法,将$\triangle r$分解为两个矩阵
\begin{equation}
    \triangle r = p_u^T q_i
\end{equation}
\begin{enumerate}
\item $p_u$ 与用户$u$相关的k维矩阵
\item $p_i$ 与电影$i$相关的k维矩阵
\end{enumerate}
\par 最终的估计模型:
\begin{equation}
    \hat{r}_{ui} = b_{ui} + p_u^T q_i
\end{equation}




\end{document}

